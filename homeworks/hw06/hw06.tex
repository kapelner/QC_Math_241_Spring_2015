\documentclass[12pt]{article}

\include{preamble}

\newtoggle{spacingmode}
\toggletrue{spacingmode}  %STUDENTS: DELETE or COMMENT this line

\newtoggle{professormode}
\toggletrue{professormode} %STUDENTS: DELETE or COMMENT this line

\newcommand{\spc}[1]{\iftoggle{spacingmode}{\\ \vspace{#1cm}}}



\title{MATH 241 Spring 2015 Homework \#6}

\author{Professor Adam Kapelner} %STUDENTS: write your name here

\iftoggle{professormode}{
\date{Due 5PM, Tuesday, Mar 24, 2015 \\ \vspace{0.5cm} \small (this document last updated \today ~at \currenttime)}
}

\renewcommand{\abstractname}{Instructions and Philosophy}




\begin{document}
\maketitle

\iftoggle{professormode}{
\begin{abstract}
The path to success in this class is to do many problems. Unlike other courses, exclusively doing reading(s) will not help. Coming to lecture is akin to watching workout videos; thinking about and solving problems on your own is the actual ``working out''.  Feel free to \qu{work out} with others; \textbf{I want you to work on this in groups.}

Reading is still \textit{required}. But for this homework set, read the relevant parts of Chapter 4 on the bernoulli, hypergeometric and binomial r.v.'s.

The problems below are color coded: \ingreen{green} problems are considered \textit{easy} and marked \qu{[easy]}; \inorange{yellow} problems are considered \textit{intermediate} and marked \qu{[harder]}, \inred{red} problems are considered \textit{difficult} and marked \qu{[difficult]}, \inpurple{purple} problems are extra credit. The \textit{easy} problems are intended to be ``giveaways'' if you went to class. Do as much as you can of the others; I expect you to at least attempt the \textit{difficult} problems.

This homework is worth 100 points but the point distribution will not be determined until after the due date. Late homework will be penalized 10 points per day.

Between 1--15 points are arbitrarily given as a bonus (conditional on quality) if the homework is typed using \LaTeX. Links to instaling \LaTeX~and program for compiling \LaTeX~is found on the syllabus. You are encouraged to use \url{overleaf.com} (make sure you upload both the hwxx.tex and the preamble.tex file). If you are handing in homework this way, read the comments in the code; there are two lines to comment out and you should replace my name with yours and write your section. If you are asked to make drawings, you can take a picture of your handwritten drawing and insert as a figure or leave space using the \qu{$\backslash$vspace} command and draw them in after printing or attach them stapled.

The document is available with spaces for you to write your answers. If not using \LaTeX, print this document and write in your answers. \textbf{Handing it in without this printout is NOT ACCEPTABLE.} You will have to redo it. 

Keep this page printed for your records. Write your name and section below where section A is if you're registered for the 9:15AM--10:30AM lecture and section B is if you're in the 12:15PM-1:30PM lecture.

\end{abstract}

\thispagestyle{empty}
\vspace{1cm}
NAME: \line(1,0){250} ~~SECTION (A or B): \line(1,0){35}
\pagebreak
}


\iftoggle{professormode}{
\paragraph{The Binomial} All binomial, all the time... \\ \\
}



\problem We will look at hypergeometric distributions with large $N$. If $N$ is really large, sampling without replacement can be approximated by sampling with replacement. In the limit, it is sampling with replacement.

\begin{enumerate}

\easysubproblem We will now begin deriving the binomial in pieces. Parameterize a hypergeometric by setting $K = pN$. What is the parameter space for $p$?

\beqn
p \in \braces{\frac{1}{N}, \frac{2}{N}, \ldots, \frac{N - 1}{N}} 
\eeqn

\easysubproblem Write the PMF $p(x)$ for this r.v. using the $p$ parameterization using $x$ as the free variable.

\beqn
p(x) = \frac{\displaystyle \binom{pN}{x} \binom{(1 - p) N}{n - x} }{\displaystyle \binom{N}{n}}
\eeqn

\easysubproblem What limit do we take and why are we taking this limit? \\

We are taking the $\lim_{N \rightarrow \infty}$ because we are trying to derive the PMF of sampling $n$ successes or failures with replacement.

\easysubproblem Rewrite the PMF without choose notation using only factorials and simplify the fraction by moving the factorial terms from denominator, $\binom{N}{n}$, to the numerator.

\beqn
p(x) = \lim_{N \rightarrow \infty} \frac{(pN)!}{\inpurple{x!} (pN - x)!} \frac{((1-p)N)!}{\inpurple{(n-x)!} ((1-p)N - n + x)!} \frac{\inpurple{n!} (N-n)!}{N!} 
\eeqn

\easysubproblem Which three terms can you factor out from the limit expression? Show that they are equivalent to $\binom{n}{x}$.

Since we are taking the $\lim_{N \rightarrow \infty}$, $p$ and $n$ and $x$ are all constants. Thus, we can pull out all terms that are not functions of $N$. We can pull out the $x!$, $(n-x)!$ and $n!$ (see the terms in part d highlighted in purple). If we pull them out, we get:

\beqn
\frac{1}{x!} \times \frac{1}{(n-x)!} \times n! = \binom{n}{x}
\eeqn


\hardsubproblem Within the limit, you now have three ratios. Write these ratios by canceling out the common terms. For instance $10!/6! = 10 \times 9 \times 8 \times 7$ and $6!/10! = 1 / (10 \times 9 \times 8 \times 7)$. This is difficult because you have to get the indexing right.  \spc{6}

\intermediatesubproblem How many terms are in the numerator? How many terms are in the denominator.  \spc{2}

\intermediatesubproblem Reason in English that the denominator looks like a bunch of $N - c_i$ where the $c_i$'s are all constants which are negligible as $N \rightarrow \infty$. \spc{3}

\intermediatesubproblem Reason in English that the numerator looks like a bunch of $Np - c_i$ where the $c_i$'s are all constants which are negligible as $N \rightarrow \infty$ as well as a bunch of $N(1-p) - c_i$ where the $c_i$'s are all constants which are negligible as $N \rightarrow \infty$.  \spc{3}

\intermediatesubproblem Match each $Np - c_i$ term in the numerator to one $N - c_i$ term in the denominator and take the limit of each one individually. Show that you wind up with $p \times p \times \ldots$ for a total of $x$ times, i.e. $p^x$. \spc{4}

\intermediatesubproblem Match each $N(1-p) - c_i$ term in the numerator to one $N - c_i$ term in the denominator and take the limit of each one individually. Show that you wind up with  $(1-p) \times (1-p) \times \ldots$ for a total of $n-x$ times, i.e. $ \tothepow{1-p}{n-x}$. \spc{4}

\easysubproblem Using your answers from parts (i), (n) and (o) write the binomial r.v.'s PMF. \spc{2}

\extracreditsubproblem Imagine you are drawing $n$ successes with and without replacement. Derive a function of $N$ and $n$ which gives this percentage difference for $N$ and $n$ generally when the number of successes $K = \half N$. \spc{7}

\end{enumerate}



\problem We will look at sums of Bernoullis and derive the Binomial through this angle.


\end{document}